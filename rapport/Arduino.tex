\section{Explications}

Le rôle de l'Arduino est d'envoyer des données au serveur pour qu'il les affiche sur la
page web, et de recevoir les données entrées sur la page web par l'utilisateur pour contrôler
la voiture. Elle récupère ces données grâce à des capteurs (capteurs de luminosité et
de température). La communication se fait par nRF24.

\section{Fichiers à inclure}
\begin{DDbox}{\linewidth}
\begin{lstlisting}
        #include <SPI.h>
	#include <RF24_config.h>
	#include <RF24.h>
	#include <nRF24L01.h>
	#include <printf.h>
	#include <Servo.h>
\end{lstlisting}
\end{DDbox}

\section{Variables}

Nous avons besoin de configurer 5 pins pour :
\begin{itemize}
	\item Le moteur droit
	\item Le moteur gauche
	\item Le capteur de luminosité
	\item Le capteur de température
\end{itemize}
\bigbreak

\begin{DDbox}{\linewidth}
\begin{lstlisting}
	// Pin configuration
	int leftM = 6; // left motor
	int rightM = 7; // right motor
	int lightSensor = 0;
	int tempSensor = 1;

\end{lstlisting}
\end{DDbox}

Ensuite, nous devons définir les variables qui contiendront les commandes que reçoit
l'Arduino :
\begin{itemize}
	\item Vitesse du moteur droit
	\item Vitesse du moteur gauche
\end{itemize}

Ainsi qu'une structure dédiée à la réception du message des commandes
de la voiture. Ce qui donne le code suivant :

\bigbreak
\begin{DDbox}{\linewidth}
\begin{lstlisting}
	// Motor command variables
	int leftSpeed;
	int rightSpeed;
	struct _motorCmd {
		int speed;
		int steer;
	} motorCmd;

\end{lstlisting}
\end{DDbox}

De la même manière, on définit une structure dédiée à l'envoi des messages
de l'Arduino au serveur.

\bigbreak
\begin{DDbox}{\linewidth}
\begin{lstlisting}
	// Sensors handling
	enum {TEMP, LIGHT};
	struct _sensorMsg {
		uint16_t type;
		float value;
	} sensorMsg;

\end{lstlisting}
\end{DDbox}

Des timers seront nécessaires car nous envoyons des données en continue, et nous devons
savoir quand il faut émettre (par exemple, toutes les secondes, ou toutes les millisecondes)
pour ne pas surcharger le serveur. Ici, nous allons éméttre la température toutes les secondes
et la luminosité toutes les millisecondes (tempDelay et lightDelay). 
Deux autres variables sont nécessaires pour tester si le délai de la température ou de la
luminosité se sont écoulés (timerTemp et timerLight).

\bigbreak
\begin{DDbox}{\linewidth}
\begin{lstlisting}
	// Timers
	int tempDelay = 1000;
	int lightDelay = 100;
	int timerTemp, timerLight;
	int time;

\end{lstlisting}
\end{DDbox}

On configure la nRF :

\bigbreak
\begin{DDbox}{\linewidth}
\begin{lstlisting}
	// nrf configuration
	int nrfCEpin = 40;
	int nrfCSpin = 53;
	uint8_t addresses[][6] = {"meteo", "motor"};
	RF24 radio = RF24(nrfCEpin, nrfCSpin);

\end{lstlisting}
\end{DDbox}


Enfin, on définit des variables de types Servo pour associer les pins
aux moteurs :
\bigbreak
\begin{DDbox}{\linewidth}
\begin{lstlisting}
	Servo rightWheel;
	Servo leftWheel;
\end{lstlisting}
\end{DDbox}


\section{Fonction setup()}

La fonction setup() va initialiser les pipes de la nRF en lecture ou en écriture, 
les timers, associe les pins aux servo moteurs et donne la vitesse initiale des moteurs droit et gauche. 

\bigbreak
\begin{DDbox}{\linewidth}
\begin{lstlisting}
	void setup() {
		Serial.begin(9600);
		radio.begin();
		radio.openWritingPipe(0x0000000001LL);
		radio.openReadingPipe(1,0x0000000002LL);
		radio.startListening();
		time = millis();
		timerTemp = time;
		timerLight = time;
  
		leftWheel.attach(7);
		rightWheel.attach(6);

		leftWheel.write(92);
		rightWheel.write(92);
	}
\end{lstlisting}
\end{DDbox}



\section{Fonction loop()}

La fonction loop va lire les données reçues sur la nRF s'il y en a (radio.available()), afficher 
les valeurs de vitesse et de virage reçues (Serial.print()) puis écrire ces données dans les variables correspondantes.

\bigbreak
\begin{DDbox}{\linewidth}
\begin{lstlisting}
	void loop() {
		Serial.println("BEFORE READ");
		radio.read(&motorCmd,sizeof(struct _motorCmd));
		Serial.print("Motor message : SPEED=");
		Serial.print(motorCmd.speed);
		Serial.print("; STEER=");
		Serial.println(motorCmd.steer);
		// map speed
		if(motorCmd.speed > 50){
			leftSpeed  = map(motorCmd.speed, 51, 100, 91, 0);
			rightSpeed = map(motorCmd.speed, 51, 100, 93, 180);
		}
		else if(motorCmd.speed < 50){
			leftSpeed  = map(motorCmd.speed, 49, 0, 93, 180);
			rightSpeed = map(motorCmd.speed, 49, 0, 91, 0);
		}
		else{
			leftSpeed  = 92;
			rightSpeed = 92;
		}
		// map steering
		if(motorCmd.steer < 50)
			leftSpeed  += map(motorCmd.steer, 49, 0, 1, 30);
		else if(motorCmd.steer > 50)
			rightSpeed -= map(motorCmd.steer, 51, 100, 1, 30);
    
		// right to pins
		leftSpeed  = constrain(leftSpeed , 0, 180);
		rightSpeed = constrain(rightSpeed, 0, 180);
		leftWheel.write(leftSpeed);
		rightWheel.write(rightSpeed);
	}

\end{lstlisting}
\end{DDbox}

Ensuite, l'Arduino va lire les valeurs sur les pins associés à la température ou à la
luminosité (analogRead), construire le message (remplissage des champs type et value
de la structure sensorMsg) et envoyer le message sur la nRF (radio.write()).

Pour cela, il faut enregistrer le temps qui s'est écoulé depuis le lancement du programme
et tester si on peut émettre, c'est-à-dire si le délai pour la température ou pour la
luminosité s'est écoulé. Par exemple pour la température, on enregistre le temps avant 
d'émettre (variable time), et après avoir émis (timerTemps). 
Si la différence entre time et timerTemp est supérieure au délai tempDelay, ça signifie qu'on
peut émettre.\\

On procède de la même manière pour l'envoi de la luminosité.

\bigbreak
\begin{DDbox}{\linewidth}
\begin{lstlisting}
	// send sensors' data if necessary
	time = millis();
	if(time - timerTemp > tempDelay){
		radio.stopListening();
		voltage = analogRead(tempSensor);
		voltage = (float)(1023-voltage)*10000/voltage;
		sensorMsg.type = TEMP;
		sensorMsg.value = (uint16_t)(1/(log(voltage/10000)/3975+1/298.15)-273.15);
		Serial.print("Sending temperature ");
		Serial.println(sensorMsg.value);
		if(!radio.write(&sensorMsg,sizeof(struct _sensorMsg))){
			Serial.println("Message sending failed");
		}
		else
			Serial.println("Temperature sent");
		timerTemp = millis();
		radio.startListening();   
	}
	if(time - timerLight > lightDelay){
		radio.stopListening();
		sensorMsg.type = LIGHT;
		sensorMsg.value = (int)(analogRead(lightSensor));
		Serial.print("Sending light ");
		Serial.println(sensorMsg.value);
		if(!radio.write(&sensorMsg,sizeof(struct _sensorMsg))){
			Serial.println("Message sending failed");
		}
		else
			Serial.println("Light sent");
		timerLight = millis();
		radio.startListening();
	}

\end{lstlisting}
\end{DDbox}
