Le daemon est le processus permettant d'assurer les communications 
allant du serveur vers l'arduino (dans le cas où le client envoie des commandes
à la voiture) et les communications allant de l'arduino vers le serveur (envoie
des données météo prises par les capteurs). Dans ce cas, le daemon ne se 
contente pas de transmettre la valeur du capteur, il va aussi la garder dans 
un fichier, ce qui nous permet d'avoir un historique des valeurs prises par les
différents capteurs. 

Le daemon comporte donc deux threads. L'un sert à recevoir les données des
capteurs et les envoyer au serveur via des pipes. L'autre sert à recevoir les
commandes envoyées par l'utilisateur et les transmettre à la voiture.

\section{Prérequis}

Afin que le daemon fonctionne, il faut que les capteurs nRF24l01 soient
correctement branchés sur la Raspberry Pi et sur l'Arduino (voir le chapitre 
\ref{ch:montage} pour plus d'informations). De plus, afin de pouvoir contrôler
le nRF24l01, il faut installer la bibliothèque idoine sur la Raspberry Pi.

Cette bibliothèque se trouve dans le dépot git situé à l'adresse suivante : 
\texttt{https://github.com/tmrh20/RF24}. Pour l'installer, il faut exécuter les
commandes suivantes dans un terminal : \\

\begin{DDbox}{\linewidth}
\begin{lstlisting}[language=bash]
        git clone https://github.com/tmrh20/RF24
        cd RF24/
        make
        sudo make install
\end{lstlisting}
\end{DDbox}

Si on veut utiliser cette bibliothèque dans un programme \texttt{C/C++}, il faut
inclure le fichier d'en-tête suivant : \\

\begin{DDbox}{\linewidth}
\begin{lstlisting}
        #include <RF24/RF24.h>
\end{lstlisting}
\end{DDbox}

De plus il faut compiler le programme avec l'option
\textbf{-l\texttt{rf24-bcm}}.

\section{Fonctionnement du daemon}

Le code source du daemon est constitué de deux fichiers : un fichier 
\texttt{daemon.h} contenant l'inclusion des fichiers d'en-tête, la definition
des constantes utilisées et la déclaration des structures dont nous avons eu
besoin et un fichier \texttt{daemon.c} contenant le code du daemon.

\subsection{Le fichier daemon.h}

Cette partie a pour but d'expliquer le contenu du fichier \texttt{daemon.h}.

\pagebreak

\subsubsection{Fichiers inclus}

Voici la liste des fichiers d'en-tête nécessaire au fonctionnement du daemon : \\

\begin{DDbox}{\linewidth}
\begin{lstlisting}
        #include <stdlib.h>
        #include <stdio.h>   
        #include <sys/types.h>
        #include <sys/stat.h>
        #include <stdint.h>       
        #include <fcntl.h>
        #include <pthread.h>
        #include <RF24/RF24.h>
        #include <unistd.h>
        #include <sys/select.h>
        #include <time.h>
        #include <signal.h>
\end{lstlisting}
\end{DDbox}

\subsubsection{Les constantes}

Les communications entre le serveur et le daemon s'effectuent via des tubes
nommés. Il faut donc définir où se situent ces derniers, c'est pourquoi nous 
avons décidé de créer les constantes suivantes : \\

\begin{DDbox}{\linewidth}
\begin{lstlisting}
        #define LIGHT_PIPE "/tmp/lightpipe"
        #define SPEED_PIPE "/tmp/speedpipe"
        #define TURN_PIPE  "/tmp/turnpipe"
        #define TEMP_PIPE  "/tmp/temppipe"
\end{lstlisting}
\end{DDbox}

Il est possible de modifier ces constantes afin de créer les tubes nommés à un 
autre endroit. Il faut cependant bien faire attention d'effectuer ces mêmes
modification dans les scripts du serveur qui utilisent les tubes nommés.

Nous avons définit une constante de plus dans le fichier afin de limiter la
taille des fichiers de log produit par le thread prévu à cet effet. En effet,
si le daemon est amené à tourner longtemps, il se peut que les fichiers de logs
deviennent très gros. La constante \texttt{MAX\_VALUES} permet de définir un
nombre maximal de valeur à garder dans ce cas là. \\

\begin{DDbox}{\linewidth}
\begin{lstlisting}
        #define MAX_VALUES 10000
\end{lstlisting}
\end{DDbox}

Par défaut, la valeur de cette constante est fixée à 10 000. Cela veut dire que
le daemon va toujours conserver que les 10 000 valeurs les plus récentes pour
les capteurs (nous verrons plus tard comment ceci est réalisé). Il est bien
entendu possible de modifier cette valeur pour obtenir des fichiers de logs plus
ou moins léger.

\pagebreak

\subsubsection{Les structures}

Le daemon utilise plusieurs structures afin de réaliser le travail qu'on lui
demande. La première est la suivante : \\

\begin{DDbox}{\linewidth}
\begin{lstlisting}
        typedef struct _meteodata {
                struct timeval time;
                int16_t data;
        } MeteoData;
\end{lstlisting}
\end{DDbox}

Cette structure représente ce qui est écrit dans les fichiers de log. Le premier
champs correspond à la date à laquelle la donnée a été reçue par le daemon via
le nRF24l01. Le second champs représente la donnée reçue.

La fréquence à laquelle les mesures sont effectuées par l'arduino différe selon
les capteurs. Il faut donc que le message émit par l'arduino contienne une 
information permettant au daemon de savoir de quel capteur provient la mesure
effectuée. C'est la raison de la présence de l'enumération suivante : \\

\begin{DDbox}{\linewidth}
\begin{lstlisting}
        enum { TEMP, LIGHT };
\end{lstlisting}
\end{DDbox}

Il faut bien entendu que cette énumération soit cohérente avec celle présente
dans le code uploadé sur l'Arduino. Il est donc évident que si on modifie une
des deux énumérations, il faut propager la modification dans l'autre.

Le message émit par l'Arduino via le nRF24l01 est représenté par la structure
suivante : \\

\begin{DDbox}{\linewidth}
\begin{lstlisting}
	typedef struct request {
		uint16_t type;
		int16_t data;
	} Request;
\end{lstlisting}
\end{DDbox}

Le champs \texttt{type} prend une des valeurs de l'énumération presentée
precedemment et permet d'indiquer au daemon de quel capteur provient la
la valeur reçue. Le second champs de la structure correspond à la valeur
prise par le capteur en question. Ici aussi il faut que cette structure 
corresponde à celle présente dans le code uploadé sur l'Arduino.

La dernière structure présente dans le fichier représente ce qui est envoyé à
l'Arduino : \\

\begin{DDbox}{\linewidth}
\begin{lstlisting}
	typedef struct cmd {
		int16_t speed;
		int16_t turn;
	} Command;	
\end{lstlisting}
\end{DDbox}

Le premier champs correspond à la nouvelle vitesse de la voiture et le second
correspond à la nouvelle direction de la voiture.

\subsection{Le fichier daemon.c}

Ce fichier contient le code du daemon.

\pagebreak

\subsubsection{Variables globales}

La première variable globale est l'objet \texttt{radio} représentant le nRF24l01
qui nous permet de communiquer avec l'Arduino. \\

\begin{DDbox}{\linewidth}
\begin{lstlisting}
        RF24 radio(22, 0);
\end{lstlisting}
\end{DDbox}

Les quatres variables globales suivantes servent à stocker les descripteurs de
fichier associés aux tubes nommées servant à communiquer avec le serveur. \\

\begin{DDbox}{\linewidth}
\begin{lstlisting}
        int fdspeed, fdturn, fdtemp, fdlight;
\end{lstlisting}
\end{DDbox}

Les variables globales \texttt{templog} et \texttt{lightlog} permettent de 
conserver les descripteurs de fichiers associés aux fichiers de log. \\

\begin{DDbox}{\linewidth}
\begin{lstlisting}
        int templog, lightlog;
\end{lstlisting}
\end{DDbox}

Comme le programme est constitué de deux threads, il n'est pas impossible (il
est même fort probable) qu'il y ait des accès concurrents au nRF24l01. C'est 
pourquoi nous avons besoin d'une sémaphore assurant l'exclusion mutuelle dans
les passages du code où l'on utilise les fonctions d'emission/réception de
données. Ce mécanisme est déjà présent dans les threads POSIX sous la forme d'un
type : \texttt{pthread\_mutex\_t}. \\


\begin{DDbox}{\linewidth}
\begin{lstlisting}
        pthread_mutex_t radio_mutex = PTHREAD_MUTEX_INITIALIZER;
\end{lstlisting}
\end{DDbox}

Afin d'obtenir des fichiers de log ayant une taille fixe, nous utilisons la 
méthode suivante : au lieu d'écrire les valeurs directement dans le fichier
correspondant, on les écrit dans un tableau de taille \texttt{MAX\_VALUES}.
Quand on arrive à la fin du tableau, on retourne au début du tableau pour 
écrire les nouvelles valeurs. De ce fait, on a toujours les \texttt{MAX\_VALUES}
les plus récentes. L'écriture des valeurs dans le fichier s'effectue au moment 
où on interrompt le daemon. Cette méthode necessite donc un tableau par capteur,
et un entier représentant l'indice permettant de savoir où écrire la prochaine
valeur que l'on va recevoir. \\

\begin{DDbox}{\linewidth}
\begin{lstlisting}
        int temp_index = 0;
        int light_index = 0;
        MeteoData tempvalues[MAX_VALUES];
        MeteoData lightvalues[MAX_VALUES];
\end{lstlisting}
\end{DDbox}

\subsubsection{Initialisations}

Au démarrage du programme, on vérifie que le nombre d'arguments est correct. Si
ce n'est pas le cas, on stoppe le programme et on affiche un message d'erreur.
\\

\begin{DDbox}{\linewidth}
\begin{lstlisting}
	//Check arguments
	if(argc!=3){
		fprintf(stderr,"Usage : %s <templogfile> <lightlogfile>\n",argv[0]);
		exit(EXIT_FAILURE);
	}
\end{lstlisting}
\end{DDbox}

Ensuite, on change le comportement par défaut lors de la reception d'un signal
\texttt{SIGINT} (que l'on peut généralement envoyer à un programme en effectuant
un \texttt{CTRL+C} dans un terminal). Dans l'appel à la fonction
\texttt{sigaction}, on indique que lors de la réception d'un signal
\texttt{SIGINT}, on veut que la fonction \texttt{handler} soit éxecutée. Cette
partie est nécessaire pour que le programme se termine correctement lorsqu'il
reçoit un signal \texttt{SIGINT}.\\

\begin{DDbox}{\linewidth}
\begin{lstlisting}
	/*Sigaction to catch SIGINT*/
	struct sigaction sigact;
	sigact.sa_handler = handler;
	sigaction(SIGINT, &sigact, NULL);
\end{lstlisting}
\end{DDbox}

\textbf{Remarque} : On peut redéfinir le comportement d'autre signaux, si on a
l'intention de tuer le daemon avec autre chose qu'un \texttt{CTRL+C}. Cependant
cette méthode ne fonctionne pas avec le signal \texttt{SIGKILL}, il est donc
vivement déconseillé de tuer le daemon avec ce signal.

On masque ensuite tout les signaux pour que les signaux reçus soient traités par
le thread \texttt{logthread}. \\

\begin{DDbox}{\linewidth}
\begin{lstlisting}
	//Blocking all signals
	sigfillset(&mask);
	sigprocmask(SIG_SETMASK, &mask, NULL);
\end{lstlisting}
\end{DDbox}

Il faut ensuite créer les fichiers correspondant aux tubes nommés. Cette action
est réalisée par la fonction \texttt{mkfifo}, qui prend en paramètre le chemin
du tube et les droits de celui-ci. \\

\begin{DDbox}{\linewidth}
\begin{lstlisting}
	if(mkfifo(SPEED_PIPE, 0766)==-1){
		perror("mkfifo (speedpipe)");
		exit(EXIT_FAILURE);
	}
\end{lstlisting}
\end{DDbox}

Pour les tubes qui sont écrit par le serveur et lus par le daemon (c'est a dire
les tubes \texttt{SPEED\_PIPE} et \texttt{TURN\_PIPE}), il faut effectuer un 
\texttt{chmod} en plus car le \texttt{umask} (masque de création de fichier) 
de la Raspberry Pi ne nous permet pas de créer les pipes avec des droits 
d'écriture pour les autres utilisateurs. En effet, les droit des tubes 
crées par \texttt{mkfifo} sont en fait issus de l'équation suivante : 
\texttt{Droit = ARG \& umask}. Il est a noter que l'utilisateur éxecutant le 
daemon (ici root, car la bibliothèque nRF24l01 à besoin d'ouvrir
\texttt{/dev/mem}) et l'utilisateur éxecutant les scripts du serveur sont
différents, c'est pourquoi nous avons recours à ce genre de méthodes. \\

\begin{DDbox}{\linewidth}
\begin{lstlisting}
	//Chmod the pipe, otherwise the server won't be able to write it
	if(chmod(SPEED_PIPE, 0766)==-1){
	        perror("chmod (speedpipe)");
	        exit(EXIT_FAILURE);
	}
\end{lstlisting}
\end{DDbox}

Il faut ensuite ouvrir les tubes nommés que nous venons de créer. On ouvre
tout les tubes en lecture \textbf{ET} en écriture pour que l'ouverture ne soit
pas bloquante, et que programme puisse continuer sans qu'il y ait un 
interlocuteur à l'autre bout du tube. \\

\begin{DDbox}{\linewidth}
\begin{lstlisting}
	if((fdspeed=open(SPEED_PIPE, O_RDWR | O_TRUNC))==-1){
		perror("open (speedpipe)");
		exit(EXIT_FAILURE);
	}
\end{lstlisting}
\end{DDbox}

Après avoir effectué ces actions pour tout les tubes que l'on veut créer (le
chmod n'étant utile que pour les tubes écrits par le serveur), il faut 
déterminer la valeur du descripteur de fichier la plus grande parmis les 
descripteurs écrit par le serveur (c'est nécessaire pour pouvoir 
scruter plusieurs descripteurs en même temps grâce à un \texttt{select}). Ceci
est réalisé par la ligne suivante :\\

\begin{DDbox}{\linewidth}
\begin{lstlisting}
	fdmax = (fdspeed<fdturn) ? (fdturn+1) : (fdspeed+1);
\end{lstlisting}
\end{DDbox}

Ensuite, il faut initialiser le nRF24l01 et ouvrir les tubes de communication
nécessaires. Dans notre cas, on ouvre un tube en écriture pour la communication 
entre le serveur et l'arduino pour les commandes de la voiture (d'adresse 
\texttt{0x000000000002LL}), et un tube en lecture pour les communications allant
de l'arduino au serveur (d'adresse \texttt{0x000000000001LL}). \\

\begin{DDbox}{\linewidth}
\begin{lstlisting}
	//Radio initialization
	radio.begin();
	radio.openWritingPipe(0x000000000002LL);
	radio.openReadingPipe(1,0x000000000001LL);
	radio.startListening();
\end{lstlisting}
\end{DDbox}

On crée ensuite le thread \texttt{logthread} qui va logger les valeurs prises
par les capteur présents sur l'Arduino. Le premier argument de l'appel à
\texttt{pthread\_create} permet de récuperer l'identifiant du thread 
nouvellement crée. Le second permet de spécifier des options particulieres (ce
que nous n'avons pas fait). Le troisième est un pointeur sur la fonction 
éxecutée par le thread (ici \texttt{logthread}). Le dernier argument est un 
pointeur sur une zone de la mémoire contenant les arguments du thread. \\

\begin{DDbox}{\linewidth}
\begin{lstlisting}
	//Create thread
	if(pthread_create(&tid, NULL, logthread, (void *) (&argv[1]))!=0){
		perror("pthread_create");
		exit(EXIT_FAILURE);
	}
\end{lstlisting}
\end{DDbox}

\subsubsection{Le thread logthread}

La première action effectuée par le thread lors de sa création est le changement
du masque des signaux. En effet, comme le thread principal a bloqué tout les 
signaux, le thread nouvellement crée ne peut en recevoir. Il faut donc démasquer
tout les signaux, sinon l'envoie d'un \texttt{SIGINT} au processus n'entrainera
pas sa terminaison. Pour réaliser ceci, on utilise la fonction 
\texttt{pthread\_sigmask}. On souhaite que ce soit ce thread qui traite la 
récéption du signal \texttt{SIGINT} car dans le cas où c'est le thread principal
qui l'effectue, il se peut qu'il y ait des accès concurrents à certaines 
variables globales, et cela peut donc entrainer des incohérences. \\

\begin{DDbox}{\linewidth}
\begin{lstlisting}
  	//Set right mask
	sigemptyset(&mask);
	pthread_sigmask(SIG_SETMASK, &mask, NULL);
\end{lstlisting}
\end{DDbox}

Si le daemon ne s'éxecute pas pour la première fois, il faut lire les valeurs
contenues dans les fichiers de logs précédemment écrit par celui-ci. Pour 
réaliser cette opération, on récupere la structure \texttt{stat} associée au
fichier (cette structure contient les informations de celui-ci : taille, droits,
...). On ouvre ensuite le fichier en lecture et on lit son contenu dans le 
tableau idoine vu précédemment. Dans le cas ou le fichier de log contenait moins
de \texttt{MAX\_VALUES}, il faut calculer l'index de la case où la prochaine 
donnée doit être insérée. Le code correspondant est le suivant : \\

\begin{DDbox}{\linewidth}
\begin{lstlisting}
  	//Get values from files, if there are any
	if(stat(filenames[0],&st)==0){
		int fd;
		if((fd=open(filenames[0],O_RDONLY))==-1){
			perror("open templogfile (read)");
			exit(EXIT_FAILURE);
		}

		if(read(fd,tempvalues, st.st_size)!=st.st_size){
			perror("read templogfile");
			exit(EXIT_FAILURE);
		}

		temp_index=(st.st_size/sizeof(MeteoData))%MAX_VALUES;
	}
\end{lstlisting}
\end{DDbox}

Il faut ensuite ouvrir chaque fichier de log en écriture, et le créer dans le 
cas ou il n'existerai pas. \\

\begin{DDbox}{\linewidth}
\begin{lstlisting}
	//Open temperature logfile
	if((templog=open(filenames[0], O_WRONLY | O_CREAT))==-1){
		perror("open (templogfile)");
		exit(EXIT_FAILURE);
	}
\end{lstlisting}
\end{DDbox}

Le programme rentre alors dans une boucle infinie dans laquelle il va rester
tant qu'il n'a pas reçu de signal indiquant qu'il faut se terminer. Dans cette
boucle, on verifie si une donnée a été receptionné par le nRF24l01. Si c'est le
cas, on la lit. On prend le mutex avant l'appel à la fonction \texttt{available}
et on le relache après l'appel à \texttt{read} pour s'assurer qu'on est bien le
seul à utiliser le capteur. \\

\begin{DDbox}{\linewidth}
\begin{lstlisting}
	while(1){
		//if there are bytes to be read
		pthread_mutex_lock(&radio_mutex);
		if(radio.available()){
			//Get data from arduino
			radio.read(&req, sizeof(Request));
			pthread_mutex_unlock(&radio_mutex);
\end{lstlisting}
\end{DDbox}

Il faut ensuite remplir la structure qui va être écrite dans le fichier de log
(\texttt{MeteoData}). On récupere la date de récéption de la mesure grâce à la
fonction \texttt{gettimeofday}. \\

\begin{DDbox}{\linewidth}
\begin{lstlisting}
        //Fill the fields of MeteoData
	d.data=req.data;
	gettimeofday(&(d.time),NULL);
\end{lstlisting}
\end{DDbox}

On utilise le champs \texttt{type} de la structure récéptionnée pour savoir dans
quel tableau il faut insérer la valeur. On l'écrit ensuite dans le bon pipe 
afin qu'elle soit lue par le serveur. \\

\begin{DDbox}{\linewidth}
\begin{lstlisting}
        //Write in the right array
        switch(req.type){
                case TEMP:
                tempvalues[temp_index]=d;
                temp_index=(temp_index+1)%MAX_VALUES;
          
                //Send to server
                if(write(fdtemp, &(d.data), sizeof(int16_t))!=sizeof(int16_t)){
                        perror("temppipe");
                        exit(EXIT_FAILURE);
                }
                break;
\end{lstlisting}
\end{DDbox}

Dans le cas où aucune donnée n'est disponible, on relache le mutex que l'on 
avait pris auparavant, et on s'endort pendant 100ms. \\

\begin{DDbox}{\linewidth}
\begin{lstlisting}
	}else{
		pthread_mutex_unlock(&radio_mutex);
		usleep(100000);
	}
\end{lstlisting}
\end{DDbox}

\subsubsection{Le thread principal}

Le thread principal éxecute aussi une boucle infinie dans laquelle il attend
qu'une écriture ait lieu sur un des descripteurs de fichier correspondant aux
tubes écrit par le serveur. Pour réaliser cette opération, on utilise la
fonction \texttt{select} qui permet d'écouter sur plusieurs descripteurs en même
temps. L'appel à cette fonction est bloquant tant qu'aucun événement ne s'est
produit sur les descripteur que l'on examine. Afin d'indiquer à la fonction
\texttt{select} les descripteurs que l'on veut écouter, on doit utiliser une
variable de type \texttt{fd\_set} qui représente un ensemble de descripteurs. \\

\begin{DDbox}{\linewidth}
\begin{lstlisting}
	//Main loop
	while(1){
		//Initialize file descriptor set
		FD_ZERO(&active_fd);
		FD_SET(fdspeed, &active_fd);
		FD_SET(fdturn, &active_fd);

		if (select(fdmax, &active_fd, NULL, NULL, NULL) < 0){
			perror ("select");
			exit (EXIT_FAILURE);
	        }
\end{lstlisting}
\end{DDbox}

Si on sort de la fonction \texttt{select}, cela veut dire qu'un des
descripteurs de l'ensemble passé en paramètre contient une donnée à lire.
La fonction modifie l'ensemble pour y indiquer les descripteurs sur lesquels il
y a eu un evenement. A la sortie de la fonction, on doit donc vérifier pour 
chaque descripteur si un evenement s'est produit. Si c'est le cas, on lit la 
valeur présente dans le tube. \\

\begin{DDbox}{\linewidth}
\begin{lstlisting}
	//If there are data in fdturn or fdspeed
	if (FD_ISSET (fdturn, &active_fd)){
		if (read(fdturn, &turn, sizeof(int16_t)) < 0){
			perror("read");
			exit(EXIT_FAILURE);
		}
		printf("New value for turn : %d\n", turn);
	}

	if (FD_ISSET (fdspeed, &active_fd)){
		if (read(fdspeed, &speed, sizeof(int16_t)) < 0){
			perror("read");
			exit(EXIT_FAILURE);
		}
		printf("New value for speed : %d\n", speed);
	}
\end{lstlisting}
\end{DDbox}

On doit ensuite envoyer les valeurs de \texttt{speed} et \texttt{turn} à
l'arduino. On doit donc utiliser le nRF24l01, c'est pourquoi on est obligé de
prendre le mutex. On doit faire un appel à \texttt{stopListening()} car d'après
la documentation, on ne doit pas faire un appel à \texttt{write()} si on n'a pas
d'abord appelé cette fonction. \\
 
\begin{DDbox}{\linewidth}
\begin{lstlisting}
	//Sends the speed and turn data
	pthread_mutex_lock(&radio_mutex);

	radio.stopListening();

	c.speed=speed;
	c.turn=turn;
	radio.write(&c, sizeof(Command));

	radio.startListening();

	pthread_mutex_unlock(&radio_mutex);
\end{lstlisting}
\end{DDbox}

\subsubsection{La terminaison du programme}

Le programme se termine quand il reçoit un signal \texttt{SIGINT}. C'est dans
la fonction \texttt{handler} que toute la magie réside. 

On doit dans un premier temps écrire les valeurs des capteurs météo contenues 
dans les tableaux dans les fichiers idoines. Il faut bien entendu que les 
valeurs écrites dans le fichier soient dans l'ordre chronologique. Pour cela,
on utilise la valeur de l'index qui nous servait à savoir ou on pouvait écrire
la prochaine donnée. En effet, on peut aussi penser à l'index comme l'indice de 
la case la plus ancienne. Il faut donc écrire les valeurs du tableau en 
commencant par cette case. On écrit donc les cases d'indices compris entre 
l'index et la fin du tableau. On écrit ensuite les valeurs allant du début
du tableau jusqu'a l'index (non inclus). Si le tableau n'était pas plein (c'est
à dire si le champs \texttt{time} de la structure présente à l'index d'écriture
est égal à 0), on écrit directement les valeurs de 0 à l'index car elles sont 
déjà dans le bon ordre. \\

\begin{DDbox}{\linewidth}
\begin{lstlisting}
	//Write datas in files
	if(tempvalues[temp_index].time.tv_sec==0){
		if(write(templog, tempvalues, temp_index*sizeof(MeteoData))!=(sizeof(MeteoData)*temp_index)){
			perror("Write templogfile");
			exit(EXIT_FAILURE);
		}
	}else{
		if(write(templog,
			 tempvalues+temp_index, 
			 (MAX_VALUES-temp_index)*sizeof(MeteoData))
		   !=(sizeof(MeteoData)*(MAX_VALUES-temp_index))){
			perror("Write templogfile");
			exit(EXIT_FAILURE);
		}
		if(write(templog, tempvalues, temp_index*sizeof(MeteoData))!=(sizeof(MeteoData)*temp_index)){
			perror("Write templogfile");
			exit(EXIT_FAILURE);
		}
	}
\end{lstlisting}
\end{DDbox}

On ferme ensuite les fichiers et on supprime les tubes nommés.

\begin{DDbox}{\linewidth}
\begin{lstlisting}
	if (close(fdlight) == -1) {
		perror("close fdlight");
		exit(EXIT_FAILURE);
	}

	//Unlink fifo
	if (unlink(LIGHT_PIPE) == -1) {
		perror("unlink lightpipe");
		exit(EXIT_FAILURE);
	}
\end{lstlisting}
\end{DDbox}

\section{Problèmes rencontrés}

L'un des problèmes rencontrés était que le serveur ne pouvait pas écrire
dans les tubes nommés prévus à cet effet. Cela était dû au fait que les 
utilisateurs autre que le créateur du fichier n'avaient pas les droits
d'écriture sur les tubes. En effet, lors de la création d'un fichier,
les droits par défaut sont au mieux ceux de l'umask. Il faut donc faire un chmod
pour donner les droits d'écritures au serveur.

Un autre problème fut le choix de la bibliothèque pour le nRF24l01. En effet,
nous avions d'abord essayé la bibliothèque présente à l'adresse https://github.com/stanleyseow/RF24, 
sans succès. Nous avons ensuite trouvé la bonne bibliothèque (voir plus bas pour le lien).

\section{Liens utilisés}

Voici une liste des liens utilisés pour la création du daemon : 
\begin{itemize}
\item \texttt{https://github.com/tmrh20/RF24} : Dépot git contenant la 
  bibliothèque permettant de contrôler le nRF24l01. 
\item \texttt{http://tmrh20.github.io/RF24/classRF24.html} : Documentation des
  fonctions de la bibliothèque nRF24l01 utilisée.
\item \texttt{man} : Afin d'écrire le daemon, nous avons consulté les pages 
  \texttt{man} de la plupart des fonctions que nous utilisons (\texttt{mkfifo, open,
    pthread\_create, ...}).
\end{itemize}
