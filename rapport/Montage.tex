Afin de réaliser notre station météo mobile, nous avons deux montages à
réaliser :
\begin{itemize}
\item le serveur (Raspberry Pi)
\item la station mobile (Arduino motorisée)
\end{itemize}

\section{Serveur}
Le serveur n'a besoin que d'un émetteur/récepteur radio nRF24l01 pour
fonctionner. De plus, il est nécessaire d'être connecté à internet pour être
accessible de l'extérieur.

\begin{center}
\includegraphics[scale=1]{include/raspberry_bb.pdf}
\end{center}


\textbf{Code couleur du cablage du nRF24l01}\\
\begin{center}
\begin{tabular}{l|c}
Couleur & Broche \\
\hline
Jaune  & CE\\
Vert   & CS\\
Bleu   & SCK\\
Violet & MISO\\
Gris   & MOSI\\
Rouge  & VCC\\
Noir   & GND
\end{tabular}
\end{center}
\pagebreak
\section{Station mobile}
Pour la réalisation de la partie mobile, nous aurons besoin :
\begin{itemize}
\item d'un Arduino (Mega dans notre cas),
\item de 2 moteurs (des servos-moteurs ici, mais vous pouvez utiliser d'autres
types de motorisations, comme un moteur de propulsion et un servo-moteur de 
direction par exemple),
\item une thermistance,
\item une résistance de 220$\Omega$,
\item un capteur de température,
\item un nRF24l01,
\item une alimentation 5V (des piles dans notre cas)
\end{itemize}

\paragraph{}
Le montage sera le suivant :
\begin{center}
\includegraphics[scale=0.7]{include/arduino_bb.pdf}
\end{center}

\paragraph{}
Le code couleur pour la connectique du nRF24l01 est la même que précédemment
pour la Raspberry. De plus, on ne parlera pas ici de l'aspect mécanique 
(chassis, placement des moteurs, etc...). En effet, de nombreux tutoriels sur
Internet détaillent bien mieux que nous ne le pourions la fabrication de la
voiture.
